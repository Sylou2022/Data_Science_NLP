pca = PCA(n_components=20)
data = pca.fit_transform(df_tfidf)
plt.plot(range(1, pca.n_components_+1), pca.explained_variance_ratio_, 'o-', linewidth=2, color='blue')
plt.xlabel('Nombre de composantes')
plt.ylabel('Variance expliquée')
plt.show()
...................................


df_tfidf = pd.DataFrame(X.toarray(), columns=feature_names)

# Effectuer le clustering avec KMeans
kmeans = KMeans(n_clusters=3)
clusters = kmeans.fit_predict(df_tfidf)

# Réduire la dimensionnalité avec t-SNE
tsne = TSNE(n_components=2, perplexity=50, random_state=42)
X_tsne = tsne.fit_transform(df_tfidf)

# Afficher le graphique des clusters
plt.scatter(X_tsne[:, 0], X_tsne[:, 1], c=clusters)
plt.show()

...............................................
import numpy as np
import matplotlib.pyplot as plt

# Calculer la somme cumulée de la variance expliquée
cumulative_variance = np.cumsum(pca.explained_variance_)

# Tracer le graphique avec la somme cumulée de la variance expliquée
plt.bar(range(1,len(pca.explained_variance_)+1), cumulative_variance)
plt.xlabel('PCA Feature')
plt.ylabel('Cumulative Explained Variance')
plt.title('Cumulative Feature Explained Variance')
plt.show()
......................................................
pour la boucle de tests des PCA:
y=[]
variances = []
for n in range(2, 12):
    pca = PCA(n_components=n)
    X_pca = pca.fit_transform(X.toarray())

    variances.append(pca.explained_variance_ratio_)
    y.append(np.sum(pca.explained_variance_ratio_))
........................................................
pour le graphe où on estsensé avoir un coude:
plt.figure(figsize=(10, 6))
plt.plot(range(2, len(variances)+2), y, '-o', label="Génération ")
plt.xlabel("génération")
plt.ylabel("Variance expliquée")
plt.legend()
plt.show()
print(pca.explained_variance_ratio_)